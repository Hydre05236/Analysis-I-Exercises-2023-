\documentclass{article}
\usepackage{graphicx} % Required for inserting images
\usepackage{ctex}
\usepackage{amsmath, amsthm, amssymb, bm, color, framed, graphicx, hyperref, mathrsfs}
\usepackage{geometry}
\geometry{left=2cm,right=2cm,top=2cm,bottom=2cm}
\everymath{\displaystyle}
\title{\textbf{周周练}}
\author{Clarigatio}
\date{\today}
\linespread{1.5}
\definecolor{shadecolor}{RGB}{241, 241, 255}
\newcounter{problemname}
\newenvironment{problem}{\begin{shaded}\stepcounter{problemname}\par\noindent\textbf{Problem }}{\end{shaded}\par}
\newenvironment{solution}{\par\noindent\textbf{解答. }}{\par}
\newenvironment{lemma}{\par\noindent\textbf{引理. }}{\par}
\newenvironment{lemmaproof}{\par\noindent\textbf{引理的证明. }}{\par}
\newenvironment{note}{\par\noindent\textbf{注记. }}{\par}
\begin{document}
\maketitle
\tableofcontents
\newpage
\section*{前言}
对周周练中的内容/期末考试复习/后续数学课程学习有疑问,或有意向参加大学生数学竞赛的同学,可以加我的QQ:422276215私聊.
\newpage
\section{题目部分}
\subsection{第四次周练}
\begin{problem}{A}\par
(A1)定义$f(x)=\ln(x+\sqrt{x^2+1})$,其中$x\ge 1$.求$f(x)$的反函数.\par
(A2)计算极限$\lim_{x\rightarrow 0^+} \left( \frac{1}{x^2} \right) ^{\tan x}$.\par
(A3)设$x_1=1,x_n=1+\frac{x_{n-1}}{1+x_{n-1}}$,证明$\lim_{n\to\infty}x_n$存在,并求此极限.\
\end{problem}
\begin{problem}{B}\par
(B1)设$f(x)$在$(0,+\infty)$上满足函数方程$f(2x)=f(x)$,且$\lim_{x\to+\infty}f(x)=A$.证明:$f(x)\equiv A$, $x\in(0,+\infty)$.\par
(B2)若函数$f(x)$在$[a,+\infty)$上连续,且$\lim_{x\to+\infty}f(x)=A$(有限数),则$f(x)$在$[a,+\infty)$上一致连续.
\end{problem}
\begin{problem}{C}\par
证明:$f(x)$在区间$I$上一致连续当且仅当对任意的$\varepsilon>0$,存在$l>0$使得当$x_1\ne x_2\in I$,且
$$
\left| \frac{f\left( x_1 \right) -f\left( x_2 \right)}{x_1-x_2} \right|>l
$$
蕴含$|f(x_1)-f(x_2)|<\varepsilon$.
\end{problem}
\newpage
\subsection{第五次周练}
\begin{problem}{A}\par
(A1)已知函数$f(x)$满足$\lim_{x\rightarrow 0} \frac{\sqrt{1+f\left( x \right) \sin 2x}-1}{e^{3x}-1}=2$,求$\lim_{x\to 0}f(x)$.
\par
(A2)设函数$f\left( x \right) =\lim_{n\rightarrow \infty} \frac{1+x}{1+x^{2n}}$,试求$f(x)$的间断点,并判断其类型.\par
(A3)设$f(x)$在$[a,b]$上连续,$a<c<d<b$.试证明:对任意的正数$p,q$,至少存在$\xi\in[c,d]$使$pf\left( c \right) +qf\left( b \right) =\left( p+q \right) f\left( \xi \right) $.\par
\end{problem}
\begin{problem}{B}\par
(B1)设$f(x)=a_1\sin x+a_2\sin 2x+\cdots+a_n\sin nx$,其中$a_1,a_2,\cdots,a_n\in\mathbb{R}$,$n$是正整数.已知对一切实数$x$有$|f(x)|\le|\sin x|$,证明:$|a_1+2a_2+\cdots+na_n|\le 1$.\par
(B2)设$f(x)$是可导函数,证明:(I)若$f(x)$为偶(奇)函数,则$f^\prime(x)$为奇(偶)函数; (II)若$f(x)$是以$T$为周期的函数,则$f^\prime(x)$也是以$T$为周期的函数.
\end{problem}
\begin{problem}{C}\par
设$\delta>0$,实数列$\{x_n\}$满足$x_{n+1}=x_n\left(1-\frac{h_n}{n^\alpha}\right)+\frac{1}{n^{\alpha+\delta}}$,这里$\{h_n\}$有正的上下界,证明$\{n^\delta x_n\}$有界.
\end{problem}
\newpage
\subsection{第六次周练}
\begin{problem}{A}\par
(A1)设$f\left( x \right) =\left( x-2 \right) \sqrt[3]{\left( x+1 \right) ^2\left( x-1 \right)}$,求$f^\prime(2)$.\par
(A2)设$f(x)$连续,且$\lim_{x\rightarrow 0} \frac{3-f\left( x+1 \right)}{2x}=-1$,求曲线$y=f(x)$在$x=1$处的切线和法线方程.\par
(A3)求(1)$y=\arcsin\frac{x}{2}$;\hspace{0.5cm}(2)$y=e^{\cos ^2\frac{1}{x}}$的导数.
\end{problem}
\begin{problem}{B}\par
(B1)设$f(x),g(x)$都在$[a,b]$上连续,在$(a,b)$内可导,且$g(x)\ne 0$, $f(a)g(b)=g(a)f(b)$,试证至少存在一点$\xi\in (a,b)$,使得$f^\prime(\xi)g(\xi)=f(\xi)g^\prime(\xi)$.\par
(B2)设$f(x),g(x),h(x)$均在$[a,b]$上连续,在$(a,b)$内可导.证明存在一点$\xi\in (a,b)$,使得
$$
\left| \begin{matrix}
	f\left( a \right)&		g\left( a \right)&		h\left( a \right)\\
	f\left( b \right)&		g\left( b \right)&		h\left( b \right)\\
	f^{\prime}\left( \xi \right)&		g^{\prime}\left( \xi \right)&		h^{\prime}\left( \xi \right)\\
\end{matrix} \right|=0.
\\
$$
\end{problem}
\begin{problem}{C}\par
设$f:[\tau_0,\tau_1]\to[0,+\infty)$, $\tau_0\ge 0$.如果存在$\theta\in[0,1)$, $A,B\ge 0$, $\alpha>0$,使得
$$
f\left( t \right) \le \theta f\left( s \right) +\frac{A}{\left( s-t \right) ^{\alpha}}+B,\hspace{0.5cm}\left( \tau _0\le t<s\le \tau _1 \right) 
$$
证明存在$C(\alpha,\theta)>0$,使得
$$
f\left( t \right) \le C\left( \alpha ,\theta \right) \left[ \frac{A}{\left( s-t \right) ^{\alpha}}+B \right] .\hspace{0.5cm}\left( \tau _0\le t<s\le \tau _1 \right) 
$$
\end{problem}
\newpage
\subsection{第七次周练}
\begin{problem}{A}\par
(A1)设函数$f(x)$在$[0,+\infty)$上可导,且$f(0)=0$, $\lim_{x\to +\infty}f(x)=2$,证明存在$a>0$使得$f(a)=1$,且存在$\xi\in(0,a)$使得$f(\xi)=\frac{1}{a}$.\par
(A2)求极限$\lim_{x\to+\infty}\left[x-x^2\ln\left(1+\frac{1}{x}\right)\right]$.\par
(A3)设$\lim_{x\rightarrow 0} \frac{\ln \left( 1+x \right) -\left( ax+bx^2 \right)}{x^2}=2$,利用Taylor公式求解$a,b$的值.
\end{problem}
\begin{problem}{B}\par
(B1)设函数$f(x)$在$[a,b]$上连续,可导且$f(a)=0$,若存在正常数$k$使得$|f^\prime(x)|\le k|f(x)|$.证明:在$[a,b]$上$f(x)$恒等于$0$.\par
(B2)设函数$f(x)$和$g(x)$在$(-1,1)$内无限次可导,且
$$|f^{(n)}(x)-g^{(n)}(x)|\le n!x,|x|<1,n=0,1,2,\cdots.$$
试证:在$(-1,1)$内$f(x)-g(x)$恒等于$0$.
\end{problem}
\begin{problem}{C}\par
设$f(x)$在$\mathbb{R}$上$m$阶可导,$m\ge 2$,记$M_k=\sup_{x\in R}|f^{(k)}(x)|$,$k=0,1,\cdots,m$,证明$M_k\le C_kM_0^{1-\frac{k}{m}}M_m^{\frac{k}{m}}$,$k=1,2,\cdots,m$,这里$C_k=\frac{k(m-k)}{2}$是一个常数.
\end{problem}
\newpage
\subsection{第八次周练}
\begin{problem}{A}\par
(A1)求$\int{\frac{\mathrm{d}x}{a^2\sin ^2x+b^2\cos ^2x}},ab\ne 0$.\par
(A2)求$\int{\frac{1}{x}\sqrt{\frac{x+2}{x-2}}\mathrm{d}x}.$\par
(A3)求$\int{\frac{xe^{-x}}{\left( 1+e^{-x} \right)}\mathrm{d}x}.$
\end{problem}
\begin{problem}{B}\par
(B1)设$R(u,v,w)$是$u,v,w$的有理函数,给出$\int R(x,\sqrt{a+x},\sqrt{b+x})\mathrm{d}x$的求法.\par
(B2)设$n$次多项式$p(x)=\sum_{i=0}^na_ix^i$,系数满足关系$\sum_{i=1}^n\frac{a_i}{(i-1)!}=0$,证明不定积分$\int p\left(\frac{1}{x}\right)e^x\mathrm{d}x$是初等函数.
\end{problem}
\begin{problem}{C}\par
设$\theta\in\mathbb{R}$,对任意的$Q\in(1,+\infty)$,存在$p,q\in\mathbb{Z}$使得$1<q<Q$,且
$$\left|\theta-\frac{p}{q}\right|\le\frac{1}{Qq}.$$
\end{problem}
\newpage
\subsection{第九次周练}
\begin{problem}{A}\par
(A1)设$G^\prime(x)=\arcsin(x-1)^2$, $G(0)=0$, 求$\int_0^1G(x)\mathrm{d}x$.\par
(A2)求$\int_0^a\sqrt{a^2-x^2}\mathrm{d}x$.\par
(A3)设$a_1+a_2+\cdots +a_n=0$,求证方程
$$
na_nx^{n-1}+\left( n-1 \right) a_{n-1}x^{n-2}+\cdots +2a_2x+a_1=0
$$
在$(0,1)$内至少有一个实根.
\end{problem}
\begin{problem}{B}\par
(B1)设$f(x)$为连续函数,且对一切$A>0$,$\int_A^{+\infty}\frac{f(x)}{x}\mathrm{d}x$收敛.证明:对任意$b>a>0$,有
$$\int_0^{+\infty}\frac{f(ax)-f(bx)}{x}\mathrm{d}x=f(0)\ln\frac{b}{a}.$$\par
(B2)设$\int_0^{+\infty}\frac{\sin x}{x}\mathrm{d}x=a$.试证:$\int_0^{+\infty}\left(\frac{\sin x}{x}\right)^2\mathrm{d}x=a$.
\end{problem}
\begin{problem}{C}\par
设$f$是定义在$\mathbb{R}$上的有界连续函数,且对任意的$x\in\mathbb{R}$,成立$f(x)=\int_{x}^{x+1}f(t)\mathrm{d}t$.证明$f$是常函数.
\end{problem}
\newpage
\section{答案部分}
\subsection{第四次周练}
\begin{problem}{A}\par
(A1)定义$f(x)=\ln(x+\sqrt{x^2+1})$,其中$x\ge 1$.求$f(x)$的反函数.\par
(A2)计算极限$\lim_{x\rightarrow 0^+} \left( \frac{1}{x^2} \right) ^{\tan x}$.\par
(A3)设$x_1=1,x_n=1+\frac{x_{n-1}}{1+x_{n-1}}$,证明$\lim_{n\to\infty}x_n$存在,并求此极限.\
\end{problem}
\begin{solution}
(A1)由函数$y=f(x)=\ln(x+\sqrt{x^2+1})$可得$e^y=x+\sqrt{x^2+1}$,整理得$x=\frac{x^{2y}-1}{2e^y}$. 所以函数$f(x)$得反函数为$f^{-1}(x)=\frac{e^{2x}-1}{2e^x}$.\par
(A2)注意到
$$
\lim_{x\rightarrow 0} \left( \frac{1}{x^2} \right) ^{\tan x}=\lim_{x\rightarrow 0} e^{\tan x\cdot \ln \frac{1}{x^2}}=\lim_{x\rightarrow 0} e^{-2\tan x\cdot \ln x}\overset{\text{洛必达法则}}{=}\lim_{x\rightarrow 0} e^{-\frac{2\sin ^2x}{x}}=e^0=1.
$$\par
(A3)显然$x_n>0$.又注意到
$$
x_n=1+\frac{x_{n-1}}{1+x_{n-1}}=2-\frac{1}{1+x_{n-1}}<2,
$$
因此数列有界.下面我们证明$\{x_n\}$是单调递增的.显然$x_2-x_1=\frac{1}{2}>0$,下面我们归纳证明$x_n>x_{n-1}$.不妨设$x_k>x_{k-1}$,那么
$$
x_{k+1}-x_k=\left( 1+\frac{x_k}{1+x_k} \right) -\left( 1+\frac{x_{k-1}}{1+x_{k-1}} \right) =\frac{x_k-x_{k-1}}{\left( 1+x_k \right) \left( 1+x_{k-1} \right)}>0,
$$
这就证明了极限的存在性.下面设$x_n\to a$,$n\to\infty$.两边取极限得$a=1+\frac{a}{1+a}$,解得$a=\frac{1+\sqrt{5}}{2}$(舍去负根).因此$\lim_{n\to\infty}x_n=\frac{1+\sqrt{5}}{2}$.
\end{solution}
\begin{problem}{B}\par
(B1)设$f(x)$在$(0,+\infty)$上满足函数方程$f(2x)=f(x)$,且$\lim_{x\to+\infty}f(x)=A$.证明:$f(x)\equiv A$, $x\in(0,+\infty)$.\par
(B2)若函数$f(x)$在$[a,+\infty)$上连续,且$\lim_{x\to+\infty}f(x)=A$(有限数),则$f(x)$在$[a,+\infty)$上一致连续.
\end{problem}
\begin{solution}
(B1)对任意的$x_0\in (0,+\infty)$,我们有
$$
f\left( x_0 \right) =f\left( 2x_0 \right) =\cdots =f\left( 2^nx_0 \right) .
$$
因此
$$
f\left( x_0 \right) =\lim_{n\rightarrow \infty} f\left( x_0 \right) =\lim_{n\rightarrow \infty} f\left( 2^nx_0 \right) =\lim_{x\rightarrow +\infty} f\left( x \right) =A,
$$
结合$x_0$的任意性就完成了证明.\par
(B2)对任给的$\varepsilon>0$,由于$\lim_{x\rightarrow +\infty} f\left( x \right) =A$,因此存在充分大的$X$,使得当$x^\prime,x^{\prime\prime}>X$时成立$|f(x^\prime)-f(x^{\prime\prime})|<\varepsilon$.由于$f$在$[a,X+1]$上连续,因此由Cantor定理知$f$在其上一致连续.现在对任给的满足$|x^\prime-x^{\prime\prime}|<\delta$的两点$x^\prime,x^{\prime\prime}\in[a,+\infty)$,不妨设$0<\delta<1$,那么必然有$|f(x^\prime)-f(x^{\prime\prime})|<\varepsilon$,这就完成了证明.
\end{solution}
\begin{problem}{C}\par
证明:$f(x)$在区间$I$上一致连续当且仅当对任意的$\varepsilon>0$,存在$l>0$使得当$x_1\ne x_2\in I$,且
$$
\left| \frac{f\left( x_1 \right) -f\left( x_2 \right)}{x_1-x_2} \right|>l
$$
蕴含$|f(x_1)-f(x_2)|<\varepsilon$.
\end{problem}
\begin{solution}
先证明一个引理.
\begin{lemma}
设$f(x)$在区间$I$上一致连续,则对任意$\varepsilon>0$,存在$M>0$,使得
$$|f(x_1)-f(x_2)|\le M|x_1-x_2|+\varepsilon,\forall x_1,x_2\in I.$$
\end{lemma}
\begin{lemmaproof}
因为$f$一致连续,所以对任意的$\varepsilon>0$.存在$\delta>0$,使得当$x,y\in I$满足$|x-y|<\delta$时成立$|f(x)-f(y)|<\varepsilon$.\par
当$|f(x)-f(y)|>\varepsilon$时,不妨设$y>x,f(y)>f(x)$,其余情形可类似讨论.令$f(y)-f(x)=kt$,这里$k\in\mathbb{N},t\in(\varepsilon,2\varepsilon]$是按如下方式取得的:$\bigcup_{k\in\mathbb{N}}(k\varepsilon,2k\varepsilon]=(\varepsilon,+\infty)$,因此要求$(k+1)\varepsilon\le 2k\varepsilon$,由此确定$k$.由介值定理,存在$x=x_0<x_1<\cdots<x_n=y$,使得$f(x_j)=f(x)+jt,j=0,1,2,\cdots,k$.进而$f(x_j)-f(x_{j-1})=t>\varepsilon,j=1,2,\cdots,k$,因此$x_j-x_{j-1}>\delta$,故$y-x>k\delta$.进而
$$
\left| f\left( x \right) -f\left( y \right) \right|=kt<\frac{t}{\delta}\left| y-x \right|\le \frac{2\varepsilon}{\delta}\left| y-x \right|,
$$
从而取$M=\frac{2\varepsilon}{\delta}$即为所求.
\end{lemmaproof}
往证原题.我们先证明充分性.如果对任意的$\varepsilon>0$,存在$l>0$,使得当$x_1\ne x_2\in I$,且$\left|\frac{f(x_2)-f(x_1)}{x_2-x_1}\right|>l$蕴含$|f(x_2)-f(x_1)|<\varepsilon$,我们取$\delta\in\left(0,\frac{\varepsilon}{l}\right)$,并取$|x_2-x_1|<\delta$,如果此时$|f(x_2)-f(x_1)|\ge\varepsilon$,那么
$$
\left| \frac{f\left( x_2 \right) -f\left( x_1 \right)}{x_2-x_1} \right|\ge \frac{\left| f\left( x_2 \right) -f\left( x_1 \right) \right|}{\delta}\ge \frac{\varepsilon}{\delta}>l\Rightarrow \left| f\left( x_2 \right) -f\left( x_1 \right) \right|<\varepsilon ,
$$
这是一个矛盾!\par
下面证明必要性.我们利用上面证明的引理,即存在$M,A>0$使得$|f(x_2)-f(x_1)|\le M|x_2-x_1|+A$.令$l=M+\frac{A}{\delta}$,当$\left|\frac{f(x_2)-f(x_1)}{x_2-x_1}\right|>l$时我们有
$$
M+\frac{A}{\left| x_2-x_1 \right|}\ge \left| \frac{f\left( x_2 \right) -f\left( x_1 \right)}{x_2-x_1} \right|>l=M+\frac{A}{\delta}\Rightarrow \left| x_1-x_2 \right|<\delta ,
$$
此时$|f(x_1)-f(x_2)|<\varepsilon$,这就完成了证明.
\end{solution}
\begin{note}
关于一致连续性的一些二级结论,读者可以参考周民强《数学分析习题演练》第一册3.3.3一节中的习题.
\end{note}
\subsection{第五次周练}
\begin{problem}{A}\par
(A1)已知函数$f(x)$满足$\lim_{x\rightarrow 0} \frac{\sqrt{1+f\left( x \right) \sin 2x}-1}{e^{3x}-1}=2$,求$\lim_{x\to 0}f(x)$.
\par
(A2)设函数$f\left( x \right) =\lim_{n\rightarrow \infty} \frac{1+x}{1+x^{2n}}$,试求$f(x)$的间断点,并判断其类型.\par
(A3)设$f(x)$在$[a,b]$上连续,$a<c<d<b$.试证明:对任意的正数$p,q$,至少存在$\xi\in[c,d]$使$pf\left( c \right) +qf\left( b \right) =\left( p+q \right) f\left( \xi \right) $.\par
\end{problem}
\begin{solution}
(A1)注意到
$$
\lim_{x\rightarrow 0} \frac{\sqrt{1+f\left( x \right) \sin 2x}-1}{e^{3x}-1}=\lim_{x\rightarrow 0} \frac{\frac{1}{2}f\left( x \right) \sin 2x}{3x}=\lim_{x\rightarrow 0} \frac{\frac{1}{2}f\left( x \right) \cdot 2x}{3x}=\frac{1}{3}\lim_{x\rightarrow 0} f\left( x \right) =2.
$$\par
(A2)先求极限:当$|x|<1$时,$\lim_{n\rightarrow \infty} \frac{1+x}{1+x^{2n}}=1+x$;当$|x|>1$时,$\lim_{n\rightarrow \infty} \frac{1+x}{1+x^{2n}}=0$;当$x=1$时$\lim_{n\rightarrow \infty} \frac{1+x}{1+x^{2n}}=1$;当$x=-1$时,$\lim_{n\rightarrow \infty} \frac{1+x}{1+x^{2n}}=0$.因此
$$
f\left( x \right) =\begin{cases}
	0,x\le -1\\
	1+x,-1<x<1\\
	1,x=1\\
	0,x>1\\
\end{cases}
$$
这是一个分段函数,分界点为$\pm 1$.由于$\lim_{x\rightarrow -1^-} f\left( x \right) =\lim_{x\rightarrow -1^+} f\left( x \right) =f\left( -1 \right) $,因此$x=-1$是$f(x)$的连续点.又$2=\lim_{x\rightarrow 1^-} f\left( x \right) \ne \lim_{x\rightarrow 1^+} f\left( x \right) =0$,因此$x=1$是$f(x)$的跳跃间断点.\par
(A3)由题意可知,$f(x)$在$[a,b]$上连续,则该区间上$f$可以取到最大值$M$和最小值$m$.进而
$$
m=\frac{pm+qm}{p+q}\le \frac{pf\left( c \right) +qf\left( b \right)}{p+q}\le \frac{pM+qM}{p+q}=M,
$$
由连续函数的介值定理知,至少存在一个$\xi\in[c,d]$使得$f\left( \xi \right) =\frac{pf\left( c \right) +qf\left( b \right)}{p+q}$,即$pf\left( c \right) +qf\left( b \right) =\left( p+q \right) f\left( \xi \right) $.
\end{solution}
\begin{problem}{B}\par
(B1)设$f(x)=a_1\sin x+a_2\sin 2x+\cdots+a_n\sin nx$,其中$a_1,a_2,\cdots,a_n\in\mathbb{R}$,$n$是正整数.已知对一切实数$x$有$|f(x)|\le|\sin x|$,证明:$|a_1+2a_2+\cdots+na_n|\le 1$.\par
(B2)设$f(x)$是可导函数,证明:(I)若$f(x)$为偶(奇)函数,则$f^\prime(x)$为奇(偶)函数; (II)若$f(x)$是以$T$为周期的函数,则$f^\prime(x)$也是以$T$为周期的函数.
\end{problem}
\begin{solution}
(B1)注意到$a_1+2a_2+\cdots+na_n=f^\prime(0)$,我们只要证明$|f^\prime(0)|\le 1$.而注意到
$$
\left| f^{\prime}\left( 0 \right) \right|=\left| \lim_{x\rightarrow 0} \frac{f\left( x \right) -f\left( 0 \right)}{x-0} \right|=\lim_{x\rightarrow 0} \left| \frac{f\left( x \right)}{x} \right|\le \lim_{x\rightarrow 0} \left| \frac{\sin x}{x} \right|=1,
$$
这就完成了证明.\par
(B2)设$f(x)$是偶函数,则$f(-x)=f(x)$.在等式两边对$x$求导,就有$f^\prime(-x)\cdot(-x)^\prime=f^\prime(x)$,即$f^\prime(x)=-f^\prime(-x)$,因此$f^\prime(x)$是奇函数.类似可以证明其他情形.
\end{solution}
\begin{note}
在(B2)中之所以可以在$f(-x)=f(x)$两边直接求导,是因为这个式子对一般的$x$都成立.对于初学者而言一个更保险的做法是利用导数的定义(同样的,我们仅证偶函数的情形):
$$
f^{\prime}\left( -x \right) =\lim_{\Delta x\rightarrow 0} \frac{f\left( -x+\Delta x \right) -f\left( -x \right)}{\Delta x}=\lim_{\Delta x\rightarrow 0} \frac{f\left( x-\Delta x \right) -f\left( x \right)}{\Delta x}=\lim_{\Delta x\rightarrow 0} \frac{-\left[ f\left( x-\Delta x \right) -f\left( x \right) \right]}{-\Delta x}=-f^{\prime}\left( x \right) .
$$
\end{note}
\begin{problem}{C}\par
设$\delta>0$,实数列$\{x_n\}$满足$x_{n+1}=x_n\left(1-\frac{h_n}{n^\alpha}\right)+\frac{1}{n^{\alpha+\delta}}$,这里$\{h_n\}$有正的上下界,证明$\{n^\delta x_n\}$有界.
\end{problem}
\begin{solution}
因为$h_n$有正的上下界,所以我们可以找到$N_2$,使得$n\ge N_2$时成立
$$
0<1-\frac{h_n}{n^{\alpha}}\le 1-\frac{C}{n^{\alpha}},
$$
这里$C=\inf\{h_n\}>0$.显然
$$
\lim_{n\rightarrow \infty} \left[ \frac{n^{\alpha +\delta}}{\left( n+1 \right) ^{\delta}}-n^{\alpha}+C \right] =C,
$$
因此存在$N=\{N_1,N_2\}$,$M=\max \left\{ \frac{2}{C},\left| x_1 \right|,2^{\delta}\left| x_2 \right|,\cdots ,N^{\delta}\left| x_N \right| \right\} $,当$n=1,2,\cdots,N$时成立$n^\delta|x_n|\le M$.设$n$时命题成立,下面用归纳法证明$n+1$时成立:
$$
\left| x_{n+1} \right|\le \left| x_n \right|\left( 1-\frac{C}{n^{\alpha}} \right) +\frac{1}{n^{\alpha +\delta}}\le \frac{M}{n^{\delta}}\left( 1-\frac{C}{n^{\alpha}} \right) +\frac{1}{n^{\alpha +\delta}}=M\left( \frac{1}{n^{\alpha}}-\frac{C}{n^{\alpha +\delta}} \right) +\frac{1}{n^{\alpha +\delta}}\le \frac{M}{\left( n+1 \right) ^{\delta}},
$$
由归纳法,我们完成了证明.
\end{solution}
\newpage
\subsection{第六次周练}
\begin{problem}{A}\par
(A1)设$f\left( x \right) =\left( x-2 \right) \sqrt[3]{\left( x+1 \right) ^2\left( x-1 \right)}$,求$f^\prime(2)$.\par
(A2)设$f(x)$连续,且$\lim_{x\rightarrow 0} \frac{3-f\left( x+1 \right)}{2x}=-1$,求曲线$y=f(x)$在$x=1$处的切线和法线方程.\par
(A3)求(1)$y=\arcsin\frac{x}{2}$;\hspace{0.5cm}(2)$y=e^{\cos ^2\frac{1}{x}}$的导数.
\end{problem}
\begin{solution}
(A1) 我们使用导数的定义.注意到
$$
f^{\prime}\left( 2 \right) =\lim_{x\rightarrow 2} \frac{f\left( x \right) -f\left( 2 \right)}{x-2}=\lim_{x\rightarrow 2} \sqrt[3]{\left( x+1 \right) ^2\left( x-1 \right)}=\sqrt[3]{9}.
$$\par
(A2) 由$2x\to 0$知为了使极限是一个常数,必须有$f(x+1)\to 3$.又函数$f$连续,我们有$f(1)=3$,从而切点为$(1,3)$.计算斜率为
$$
k=f^{\prime}\left( 1 \right) =\lim_{x\rightarrow 0} \frac{f\left( x+1 \right) -f\left( 1 \right)}{x}=\lim_{x\rightarrow 0} \frac{f\left( x+1 \right) -3}{x}=-2\lim_{x\rightarrow 0} \frac{3-f\left( x+1 \right)}{2x}=2,
$$
从而切线方程:$y=2x+1$,法线方程:$y=-\frac{1}{2}x+\frac{7}{2}$.\par
(A3) 对第一个:
$$
y^{\prime}=\frac{1}{\sqrt{1+\left( \frac{x}{2} \right) ^2}}\left( \frac{x}{2} \right) ^{\prime}=\frac{1}{\sqrt{4-x^2}};
$$
对第二个:
$$
y^{\prime}=e^{\cos ^2\frac{1}{x}}\left( \cos ^2\frac{1}{x} \right) ^{\prime}=2e^{\cos ^2\frac{1}{x}}\cos \frac{1}{x}\left( -\sin \frac{1}{x} \right) \left( \frac{1}{x} \right) ^{\prime}=\frac{1}{x^2}\sin \frac{2}{x}e^{\cos ^2\frac{1}{x}}.
$$
\end{solution}
\begin{note}
注意(A1)也可以直接求导,但计算量显然比利用导数定义更大.
\end{note}
\begin{problem}{B}\par
(B1)设$f(x),g(x)$都在$[a,b]$上连续,在$(a,b)$内可导,且$g(x)\ne 0$, $f(a)g(b)=g(a)f(b)$,试证至少存在一点$\xi\in (a,b)$,使得$f^\prime(\xi)g(\xi)=f(\xi)g^\prime(\xi)$.\par
(B2)设$f(x),g(x),h(x)$均在$[a,b]$上连续,在$(a,b)$内可导.证明存在一点$\xi\in (a,b)$,使得
$$
\left| \begin{matrix}
	f\left( a \right)&		g\left( a \right)&		h\left( a \right)\\
	f\left( b \right)&		g\left( b \right)&		h\left( b \right)\\
	f^{\prime}\left( \xi \right)&		g^{\prime}\left( \xi \right)&		h^{\prime}\left( \xi \right)\\
\end{matrix} \right|=0.
\\
$$
\end{problem}
\begin{solution}
(B1) 令$F(x)=\frac{f(x)}{g(x)}$,由$f(a)g(b)=g(a)f(b)$知$F(a)=F(b)$,因此由Rolle中值定理知存在$\xi\in (a,b)$,使得$F^\prime(\xi)=0$,即
$$
\frac{f^{\prime}\left( \xi \right) g\left( \xi \right) -f\left( \xi \right) g^{\prime}\left( \xi \right)}{g^2\left( \xi \right)}=0,
$$
这就完成了证明.\par
(B2) 令
$$
F\left( x \right) =\left| \begin{matrix}
	f\left( a \right)&		g\left( a \right)&		h\left( a \right)\\
	f\left( b \right)&		g\left( b \right)&		h\left( b \right)\\
	f\left( x \right)&		g\left( x \right)&		h\left( x \right)\\
\end{matrix} \right|.
$$
那么由行列式的性质我们有$F(a)=F(b)$.从而由Rolle中值定理知存在$\xi\in (a,b)$使得$F(\xi)=0$,即证.
\end{solution}
\begin{problem}{C}\par
设$f:[\tau_0,\tau_1]\to[0,+\infty)$, $\tau_0\ge 0$.如果存在$\theta\in[0,1)$, $A,B\ge 0$, $\alpha>0$,使得
$$
f\left( t \right) \le \theta f\left( s \right) +\frac{A}{\left( s-t \right) ^{\alpha}}+B,\hspace{0.5cm}\left( \tau _0\le t<s\le \tau _1 \right) 
$$
证明存在$C(\alpha,\theta)>0$,使得
$$
f\left( t \right) \le C\left( \alpha ,\theta \right) \left[ \frac{A}{\left( s-t \right) ^{\alpha}}+B \right] .\hspace{0.5cm}\left( \tau _0\le t<s\le \tau _1 \right) 
$$
\end{problem}
\begin{solution}
我们定义序列$\{t_n\}_{n=0}^\infty$满足$t_{i+1}=t_i+(s-t)K_i$,其中$K_i$待定.令$t_0=t$.那么由$f(t)\le\theta f(s)+\frac{A}{(s-t)^\alpha}+B$我们有
$$
f\left( t \right) =\sum_{i=0}^{\infty}{\left[ \theta ^if\left( t_i \right) -\theta ^{i+1}f\left( t_{i+1} \right) \right]}\le \sum_{i=1}^{\infty}{\theta ^i\left[ \frac{A}{\left( s-t \right) ^{\alpha}K_{i}^{\alpha}}+B \right]}=\sum_{i=0}^{\infty}{\frac{\theta ^i}{K_{i}^{\alpha}}\cdot \frac{A}{\left( s-t \right) ^{\alpha}}}+B\sum_{i=0}^{\infty}{\theta ^i}.
$$
我们期望$\lim_{n\to\infty}t_n=s$(否则$\{t_n\}$可能不全落在$[\tau_0,\tau_1]$上),即$\sum_{i=0}^\infty(t_{i+1}-t_i)=s-t$,从而
$$
\sum_{i=0}^{\infty}{\left( t_{i+1}-t_i \right)}=\sum_{i=0}^{\infty}{\left( s-t \right) K_i}=\left( s-t \right) \sum_{i=0}^{\infty}{K_i}=s-t,
$$
从而$\sum_{i=0}^\infty K_i=1$.令$K_i=(1-\tau)\tau^i$,则$\sum_{i=0}^\infty(1-\tau)\tau^i=1$,这里$\tau\in(0,1)$.现在
$$
\begin{aligned}
f\left( t \right) &\le \sum_{i=0}^{\infty}{\frac{\theta ^i}{\tau ^{\alpha i}\left( 1-\tau \right) ^{\alpha}}\cdot \frac{A}{\left( s-t \right) ^{\alpha}}}+B\sum_{i=0}^{\infty}{\theta ^i}
\\
&=\sum_{i=0}^{\infty}{\frac{\theta ^i}{\tau ^{\alpha i}\left( 1-\tau \right) ^{\alpha}}\cdot \frac{A}{\left( s-t \right) ^{\alpha}}}+B\sum_{i=0}^{\infty}{\frac{\theta ^i}{\tau ^{\alpha i}\left( 1-\tau \right) ^{\alpha}}\cdot \tau ^{\alpha i}\left( 1-\tau \right) ^{\alpha}}
\\
&\le \sum_{i=0}^{\infty}{\frac{\theta ^i}{\tau ^{\alpha i}\left( 1-\tau \right) ^{\alpha}}\left[ \frac{A}{\left( s-t \right) ^{\alpha}}+B \right]},
\end{aligned}
$$
因此令
$$
C\left( \alpha ,\theta \right) =\frac{1}{\left( 1-\tau \right) ^{\alpha}}\sum_{i=0}^{\infty}{\left( \frac{\theta}{\tau ^{\alpha}} \right) ^i},\hspace{0.5cm}\tau =\frac{\theta ^{\frac{1}{\alpha}}+1}{2}
$$
即可.
\end{solution}
\begin{note}
本题具有偏微分方程背景,事实上本题是二阶椭圆算子的一个弱极值原理的一个数学分析引理.
\end{note}
\newpage
\subsection{第七次周练}
\begin{problem}{A}\par
(A1)设函数$f(x)$在$[0,+\infty)$上可导,且$f(0)=0$, $\lim_{x\to +\infty}f(x)=2$,证明存在$a>0$使得$f(a)=1$,且存在$\xi\in(0,a)$使得$f(\xi)=\frac{1}{a}$.\par
(A2)求极限$\lim_{x\to+\infty}\left[x-x^2\ln\left(1+\frac{1}{x}\right)\right]$.\par
(A3)设$\lim_{x\rightarrow 0} \frac{\ln \left( 1+x \right) -\left( ax+bx^2 \right)}{x^2}=2$,利用Taylor公式求解$a,b$的值.
\end{problem}
\begin{solution}
(A1)令$F(x)=f(x)-1$,显然其在$[0,+\infty)$上连续,且$F(0)=f(0)-1=-1<0$,$F(+\infty)=\lim_{x\to\infty}F(x)=\lim_{x\to\infty}f(x)-1=1>0$,由连续函数的广义零点定理知存在$a>0$使得$f(a)=1$.因为$f(x)$在$[0,a]$上连续,在$(0,a)$上可导,所以由Lagrange中值定理知存在$\xi\in (0,a)$使得$f^\prime(\xi)=\frac{f(a)-f(0)}{a}=\frac{1}{a}$.\par
(A2)注意到
$$
\lim_{x\rightarrow 0} \left[ x-x^2\ln \left( 1+\frac{1}{x} \right) \right] \overset{x=\frac{1}{t}}{=}\lim_{t\rightarrow 0} \left[ \frac{1}{t}-\frac{1}{t^2}\ln \left( 1+t \right) \right] =\lim_{t\rightarrow 0} \frac{t-\ln \left( 1+t \right)}{t^2}=\lim_{t\rightarrow 0} \frac{1-\frac{1}{1+t}}{2t}=\lim_{t\rightarrow 0} \frac{1}{2\left( 1+t \right)}=\frac{1}{2}.
$$\par
(A3)由Taylor公式知
$$
\lim_{x\rightarrow 0} \frac{\left( 1-a \right) -\left( \frac{1}{2}+b \right) x^2+o\left( x^2 \right)}{x^2}=2,
$$
因此$1-a=0$,$-\frac{1}{2}+b=2$,即$a=1$,$b=-\frac{5}{2}$.
\end{solution}
\begin{problem}{B}\par
(B1)设函数$f(x)$在$[a,b]$上连续,可导且$f(a)=0$,若存在正常数$k$使得$|f^\prime(x)|\le k|f(x)|$.证明:在$[a,b]$上$f(x)$恒等于$0$.\par
(B2)设函数$f(x)$和$g(x)$在$(-1,1)$内无限次可导,且
$$|f^{(n)}(x)-g^{(n)}(x)|\le n!x,|x|<1,n=0,1,2,\cdots.$$
试证:在$(-1,1)$内$f(x)-g(x)$恒等于$0$.
\end{problem}
\begin{solution}
(B1)若$k=0$,命题显然.若$k>0$,取$\delta=\frac{1}{2k}$.将$[a,b]$实行等分割,得$\Delta x_1,\Delta x_2,\cdots,\Delta x_n$.令$\Delta x_i$表示第$i$个小区间,$|\Delta x_i|$表示第$i$个小区间得长度,$|\Delta x_i|<]\delta$,对于任意的$x\in\Delta x_1$,有
$$f(x)=f(a)+f^\prime(\xi)(x-a)=f^\prime(\xi)(x-a),$$
即
$$|f(x)|\le|f^\prime(\xi)|\cdot|\Delta x_1|\le|f^\prime(\xi)|\delta\le k|f(\xi)|\delta.$$
若令$M=\max_{x\in\Delta x_1}|f(x)|$,那么$M\le\frac{M}{2}$,这导致$M=0$,因此$f$在$\Delta x_1$上的延拓是常数.同样继续延拓下去,我们有$f\equiv 0$,即证.\par
(B2)令$x=0$,我们有$|F^{(n)}(0)|\le 0$,因此$F^{(n)}(0)=0$.对任意给定的$x\in(-1,1)$我们有
$$
F\left( x \right) =F\left( 0 \right) +F^{\prime}\left( 0 \right) x+\frac{F^{\prime\prime}\left( 0 \right)}{2!}x^2+\cdots +\frac{F^{\left( n-1 \right)}\left( 0 \right)}{\left( n-1 \right) !}x^{n-1}+\frac{F^{\left( n \right)}\left( \xi \right)}{n!}x^n,
$$
即$F(x)=\frac{1}{n!}F^{(n)}(\xi)x^n$,$\xi\in(0,x)\cap(x,0)$.从而
$$
0\le \left| F\left( x \right) \right|\le \left| \frac{1}{n!}F^{\left( n \right)}\left( \xi \right) x^n \right|\le \left| x^{n+1} \right|.
$$
令$n\to\infty$,由夹逼准则即证!
\end{solution}
\begin{problem}{C}\par
设$f(x)$在$\mathbb{R}$上$m$阶可导,$m\ge 2$,记$M_k=\sup_{x\in R}|f^{(k)}(x)|$,$k=0,1,\cdots,m$,证明$M_k\le C_kM_0^{1-\frac{k}{m}}M_m^{\frac{k}{m}}$,$k=1,2,\cdots,m$,这里$C_k=\frac{k(m-k)}{2}$是一个常数.
\end{problem}
\begin{solution}
我们用数学归纳法证明.对$k=1$的情形,我们用Taylor公式有
$$
\left| f^{\prime}\left( x \right) \right|\le \left| \frac{f\left( x+h \right) -f\left( x-h \right)}{2} \right|+\frac{\frac{h^2}{2}\left| f^{\prime\prime}\left( \theta _2 \right) \right|+\frac{h^2}{2}\left| f^{\prime\prime}\left( \theta _1 \right) \right|}{2h}\le \frac{M_0}{h}+\frac{M_2h}{2}.
$$
为了得到最佳的估计,利用基本不等式,我们有
$$
\frac{M_0}{h}+\frac{M_2h}{2}\ge 2\sqrt{\frac{M_0}{2}\cdot \frac{M_2h}{2}}=\sqrt{2M_0M_2},
$$
即证.现在假设在$k<m$时命题成立,那么
$$
M_{k+1}\le 2^{\frac{k\left( m-k \right)}{2}}M_{1}^{1-\frac{k}{m}}M_{m+1}^{\frac{k}{m}},k=1,2,\cdots ,m-1.
$$
进而对$M_m$我们有
$$
M_m\le 2^{\frac{m-1}{2}}M_{1}^{\frac{1}{m}}M_{m+1}^{1-\frac{1}{m}}\le 2^{\frac{m-1}{2}}\left( 2^{\frac{m-1}{2}}M_{0}^{1-\frac{1}{m}}M_{m}^{\frac{1}{m}} \right) M_{m+1}^{1-\frac{1}{m}}=2^{\frac{m}{2}}M_{0}^{\frac{1}{m+1}}M_{m+1}^{\frac{m}{m+1}},
$$
从而
$$
M_k\le 2^{\frac{k\left( m-k \right)}{2}}M_{1}^{1-\frac{k}{m}}M_{m}^{\frac{k}{m}}\le 2^{\frac{k\left( m-k \right)}{2}}M_{0}^{1-\frac{k}{m}}\left( 2^{\frac{m}{2}}M_{0}^{\frac{1}{m+1}}M_{m+1}^{\frac{m}{m+1}} \right) ^{\frac{k}{m}}=2^{\frac{k\left( m+1-k \right)}{2}}M_{0}^{1-\frac{k}{m+1}}M_{m+1}^{k},
$$
这里$k=1,2,\cdots,m$.
\end{solution}
\newpage
\subsection{第八次周练}
\begin{problem}{A}\par
(A1)求$\int{\frac{\mathrm{d}x}{a^2\sin ^2x+b^2\cos ^2x}},ab\ne 0$.\par
(A2)求$\int{\frac{1}{x}\sqrt{\frac{x+2}{x-2}}\mathrm{d}x}.$\par
(A3)求$\int{\frac{xe^{-x}}{\left( 1+e^{-x} \right)}\mathrm{d}x}.$
\end{problem}
\begin{solution}
(A1)注意到
$$
\int{\frac{\mathrm{d}x}{a^2\sin ^2x+b^2\cos ^2x}}=\int{\frac{\mathrm{d}\left( \tan x \right)}{a^2\tan ^2x+b^2}}=\frac{1}{ab}\mathrm{arc}\tan \left( \frac{b}{a}\tan x \right) +C.
$$\par
(A2)做代换$t=\sqrt{\frac{x+2}{x-2}}$,我们有
$$
\int{\frac{1}{x}\sqrt{\frac{x+2}{x-2}}\mathrm{d}x}=\int{\frac{4t^2\mathrm{d}t}{\left( 1-t^2 \right) \left( 1+t^2 \right)}}=\int{\left( \frac{2}{1-t^2}-\frac{2}{1+t^2} \right) \mathrm{d}t}=\ln \left| \frac{1+t}{1-t} \right|-2\mathrm{arc}\tan t+C.
$$
因此
$$
\int{\frac{1}{x}\sqrt{\frac{x+2}{x-2}}\mathrm{d}x}=\ln \left| \frac{1+\sqrt{\frac{x+2}{x-2}}}{1-\sqrt{\frac{x+2}{x-2}}} \right|-2\mathrm{arc}\tan \sqrt{\frac{x+2}{x-2}}+C.
$$\par
(A3)注意到
$$
\int{\frac{xe^{-x}}{\left( 1+e^{-x} \right)}\mathrm{d}x}=\int{x\mathrm{d}\left( \frac{1}{1+e^{-x}} \right)}=\frac{x}{1+e^{-x}}-\int{\frac{\mathrm{d}x}{1+e^{-x}}}=\frac{x}{1+e^{-x}}-\ln \left( 1+e^x \right) +C.
$$
\end{solution}
\begin{problem}{B}\par
(B1)设$R(u,v,w)$是$u,v,w$的有理函数,给出$\int R(x,\sqrt{a+x},\sqrt{b+x})\mathrm{d}x$的求法.\par
(B2)设$n$次多项式$p(x)=\sum_{i=0}^na_ix^i$,系数满足关系$\sum_{i=1}^n\frac{a_i}{(i-1)!}=0$,证明不定积分$\int p\left(\frac{1}{x}\right)e^x\mathrm{d}x$是初等函数.
\end{problem}
\begin{solution}
(B1)设$t=\sqrt{a+x}$,则
$$
\int{R\left( x,\sqrt{a+x},\sqrt{b+x} \right) \mathrm{d}x}=2\int{t\cdot R\left( t^2-a,t,\sqrt{t^2-a+b} \right) \mathrm{d}t}.
$$
再令$\sqrt{t^2-a+b}=t+u$,则
\begin{small}
$$
\int{R\left( x,\sqrt{a+x},\sqrt{b+x} \right) \mathrm{d}x}=2\int{\frac{a-b-u^2}{2u^2}\cdot \frac{b-a-u^2}{2u}\cdot R\left( \left( \frac{b-a-u^2}{2u} \right) ^2-a,\frac{b-a-u^2}{2u},\frac{b-a+u^2}{2u} \right) \mathrm{d}u}
$$
\end{small}
为有理函数的积分.此时利用有理函数的积分方法即可求解.\par
(B2)设$I_k=\int\frac{e^x}{x^k}\mathrm{d}x$,则对$k\ge 2$,我们有
$$
I_k=-\frac{1}{k-1}\int{e^k\mathrm{d}\left( \frac{1}{x^{k-1}} \right)}=-\frac{1}{k-1}\frac{e^x}{x^{k-1}}+\frac{1}{k-1}\int{\frac{e^x}{x^{k-1}}\mathrm{d}x}=-\frac{1}{k-1}\frac{e^x}{x^{k-1}}+\frac{1}{k-1}I_{k-1},
$$
由此得到
$$
I_k=q_{k-1}\left( \frac{1}{x} \right) e^x+\frac{1}{\left( k-1 \right) !}\int{\frac{e^x}{x}\mathrm{d}x},
$$
其中$q_{k-1}(x)$是关于$x$的$k-1$次多项式.当$\sum_{i=1}^n\frac{a_i}{(i-1)!}=0$时,积分
$$
\begin{aligned}
\int{p\left( \frac{1}{x} \right) e^x\mathrm{d}x}&=a_0e^x+\sum_{i=1}^n{a_i\int{\frac{e^x}{x^i}\mathrm{d}x}}
\\
&=a_0e^x+\sum_{i=2}^n{a_iq_{i-1}\left( \frac{1}{x} \right) e^x}+\sum_{i=1}^n{\frac{a_i}{\left( i-1 \right) !}\int{\frac{e^x}{x}\mathrm{d}x}}
\\
&=a_0e^x+\sum_{i=2}^n{a_iq_{i-1}\left( \frac{1}{x} \right) e^x}+C
\end{aligned}
$$
为初等函数.
\end{solution}
\begin{problem}{C}\par
设$\theta\in\mathbb{R}$,对任意的$Q\in(1,+\infty)$,存在$p,q\in\mathbb{Z}$使得$1<q<Q$,且
$$\left|\theta-\frac{p}{q}\right|\le\frac{1}{Qq}.$$
\end{problem}
\begin{solution}
首先对$Q\in\mathbb{N}$的情形,我们令$\left[ 0,1 \right] =\left[ 0,\frac{1}{Q} \right) \cup \left[ \frac{1}{Q},\frac{2}{Q} \right) \cup \cdots \cup \left[ \frac{Q-1}{Q},1 \right] $.记$\delta(x)=x-[x]$,这里$[x]$是向下取整函数.此时由抽屉原理,$0,\delta \left( \theta \right) ,\delta \left( 2\theta \right) ,\cdots ,\delta \left( \left( Q-1 \right) \theta \right) ,1$中必有两个落在上述同一区间内,不妨设为$\delta(r_1\theta)$和$\delta(r_2\theta)$(对于$0$或$1$被取到的情形可以类似讨论),其中$r_1,r_i\in\{1,2,\cdots,Q-1\}$互异.注意$\delta(\cdot)$的定义,我们知道存在$s_1,s_2\in\mathbb{Z}$使得
$$|r_1\theta-s_1-(r_2\theta-s_2)|\le\frac{1}{Q}.$$
不妨设$r_1>r_2$,因此$|q\theta-p|\le\frac{1}{Q}$, $1\le q=r_1-r_2<Q$,这就是$\left|\theta-\frac{p}{q}\right|\le\frac{1}{Qq}$.现在如果$Q\notin\mathbb{N}$,此时$[Q]+1>Q$,那么存在$1\le q<[Q]+1$使得
$$\left|\theta-\frac{p}{q}\right|\le\frac{1}{([Q]+1)q}\le\frac{1}{Qq},$$
证毕.
\end{solution}
\begin{note}
这个命题具有相当深刻的数论背景,事实上这就是\textbf{Dirichlet逼近定理}.命题中$p,q\in\mathbb{Z}$可以加强为$p,q$互素,对这方面内容感兴趣的读者可以参考朱尧辰《无理数引论》1.2节内容.
\end{note}
\newpage
\end{document}
